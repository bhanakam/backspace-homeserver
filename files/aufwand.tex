\section{Aufwand}

\begin{frame}{Aufwand}
\framesubtitle{Zeit ist Geld}
Faustformel:
\begin{itemize}
\item Pro Dienst bzw. Anwendung ca. 1 Stunde pro Monat
\pause
\item Pro Maschine (auch VMs, Container, …) ca. 1 Stunde pro Monat
\end{itemize}
\end{frame}

\begin{frame}{Aufwand}
\framesubtitle{Woher kommt Aufwand?}
\begin{itemize}
\item Updates
\pause
\item Backups
\pause
\item Webanwendungen (und deren Abhängigkeiten)
\pause
\item Ausfälle (Shit happens)
\end{itemize}
\end{frame}

\begin{frame}{Aufwand}
\framesubtitle{Energie}
TDP != Stromverbrauch:
Mit Thermal Design Power wird in der Elektronikindustrie ein maximaler Wert für die thermische Verlustleistung … elektronischer Bauteile bezeichnet, auf deren Grundlage die Kühlung sowie die Stromzufuhr ausgelegt werden.
Quelle: Wikipedia
\end{frame}

\begin{frame}{Aufwand}
\framesubtitle{Energie}
Energiekosten: 29 Cent/KWh, als Dauerlast:\\
\pause
\begin{tabular}{|l|r|r|}
\hline
\textbf{Leistung} & \textbf{Jahr} & \textbf{Monat} \\
\hline
50 Watt & 127€ &  10€ \\
\hline
35 Watt & 89€ & 7.50€ \\
\hline
15 Watt & 38€ & 3.20€ \\
\hline
\end{tabular}
\pause
\begin{itemize}
\item Intel Core i3 4130T, Gigabyte H87-HD3, 400 Watt 80+ NT => ~25 Watt Idle
\pause
\item GA-D525TUD (Atom CPU!) => ~15-20 Watt Idle
\pause
\item Speedport W723V Typ B 7.0-8.5 Watt (je nach DECT/WLAN)
\end{itemize}
\end{frame}

\begin{frame}{Aufwand}
\framesubtitle{Energie}
\begin{tabular}{|l|r|r|r|l|r|}
\hline
\textbf{Platte}   & \textbf{Standby} & \textbf{Idle} & \textbf{Work} & \textbf{Garantie} & \textbf{Euro (1TB)} \\
\hline
WD Green &     0.4 &  2.5 &  3.3 & 2 Jahre  &        47€ \\
\hline
WD Red   &     0.4 &  2.3 &  3.3 & 3 Jahre  &        60€ \\
\hline
WD RE4   &       - &  5.9 &  8.6 & 5 Jahre  &        90€ \\
\hline
WD VR    &     1.1 &  4.2 &  5.8 & 5 Jahre  &       300€ \\
\hline
WD Black &     0.8 &  6.1 &  6.8 & 5 Jahre  &        73€ \\
\hline
\end{tabular}
\end{frame}
